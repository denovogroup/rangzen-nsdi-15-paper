In this paper we describe the implementation and evaluation of Rangzen, a
delay-tolerant mesh-network-based anonymous messaging system for smartphones
which enables communication in the face of wide-scale Internet and cellular
blackouts. Rangzen's design is a general one which supports any application
which can be implemented over broadcast datagrams. Rangzen prioritizes messages
based on the social network of its users in a privacy preserving manner. This
disables malicious entities who might attempt to flood the system with
undesired messages. 

Our contributions are threefold: first, we perform experiments with users
around an urban center using their actual daily mobility and social connections.
This mobility and social dataset is an important contribute in its own right,
as existing publically available datasets are rare and are an order of magnitude
smaller in scale than our dataset. Second, we confirm our prior analytical
and simulation results with real-world experiments, showing that message propagation
in Rangzen is sufficient to support several use cases and that adversarial message
propagation is severely stemmed. Finally, we examine Rangzen's capabilities in an
adversarial game environment. Or something else new, I'm not sure. But there has
to be something new and shiny here.

